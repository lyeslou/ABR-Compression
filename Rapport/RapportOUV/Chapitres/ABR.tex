\chapter{Arbres de recherches Binaires}
\section{Structure}
    \begin{minted}{ocaml} 
        type arb = Empty | Node of int * arb * arb ;;
    \end{minted}
\section{Construction}
On construit notre arbre binaire de recherche par ajout successifs en feuille à partir d'une liste reçu en paramètre : \\ 
\begin{enumerate}
    \item On construit un nouveau noeud ayant comme clé la valeur de notre élément, Empty pour les deux fils gauche et droit.
    \item Pour chaque élément de la liste on insert celui-ci à la feuille de l'arbre. 
\end{enumerate}
De cette façon l'insertion se fait toujours au niveau des feuilles et pour une liste donnée il y a qu'un seul arbre possible a construire.
\section{Recherche}
La recherche d'un élément dans l'arbre se fait de la manière suivante :\\
\begin{itemize}
    \item si la racine = vide on renvoie false 
    \item si val(racine) = élément on renvoie true 
    \item si val(racine) < élément on continue la recherche dans le sous arbre gauche sinon on continue dans le sous arbre droit.  
\end{itemize}


\section{Complexité}
La complexité de recherche est en O(h) où h est la hauteur de l'arbre\\
\begin{itemize}
    \item Dans le pire cas (liste triée) la hauteur égal n --> O(n)
    \item Dans le cas moyen (arbre équilibré) la hauteur tend vers log(n) --> O(log (n)).
\end{itemize}




