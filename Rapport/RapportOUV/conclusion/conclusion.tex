\chapter{Conclusion}

\paragraph{}
Durant ce projet, nous avons essayé d'implémenter une compression d'arbre binaire de recherche susceptible d'améliorer les performances mémoires sans trop diminuer les performances en temps de calcul.
En effet, théoriquement cette implémentation aurait dû donner des résultats assez clairs mais malheureusement, cela n'a été le cas dans nos expérimentations :
\begin{itemize}
    \item un arbre binaire de recherche standard de 40000 éléments on a une mémoire de \textit{120000} (n + 2n ref)
    \item En prenant en compte le nombre moyen de valeurs dans un noeud \textit{moy} et le nombre de noeud total \textit{nb} pour un arbre de taille 40000 on obtient \textit{moy = 4,3} et \textit{nb = 8500}.
    \item Et la taille de notre structure peut se calculer de cette manière : \textit{nbElem + nbRef + nbLabels } où :
    \begin{itemize}
        \item \textit{nbElem = nb * moy = 36550}
        \item \textit{nbRef = 2 * nb * (ref + label) = 34000} (en effet chaque arc a une référence et un label
        \item \textit{nbLabels = n * labels = 520000} où \textit{n} est le nombre d'elements et \textit{labels} sont les labels de chaque éléments soit au pire cas \textit{labels = log(nb)}
    \end{itemize}
    ce qui donne : \textit{memoire = (nb * moy) + 2 * nb (ref + label)  + log(nb) * n = 590550}
\end{itemize}
\paragraph{}
En regardant ces résultats, on se rend compte que le plus gros problème de mémoire est la façon de stocker les labels correspondant à chaque valeur. En effet, on stocke la liste entière des labels pour chaque valeur. 
Pour remédier à cela, une idée serait de ne stocker que le label du père et non toute la liste. cette solution diminuerait grandement la mémoire de notre structure car \textit{nbLabels = n} seulement ce qui donnerait un total de \textit{110550}. 